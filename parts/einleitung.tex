\chapter{Einleitung}
  \label{chapter:einleitung}

%%% Das hier ersetzen durch Ihren Text:

\section{Problemstellung}

\subsection*{Zweck}

Der Abschnitt "Problemstellung" hat das Ziel, den Leser in das Thema der Arbeit
einzuführen, die Relevanz zu verdeutlichen und spezifische Probleme zu
identifizieren, die im Rahmen der Arbeit adressiert werden sollen. Es sollen
hier noch keine konkreten Lösungsansätze vorgestellt werden. Der Umfang sollte
ein bis zwei Seiten umfassen.

\subsection*{Inhalt}

\begin{enumerate}

\item Einführung in das Thema

Beginnen Sie mit einem allgemeinen Überblick über das Themenfeld Ihrer Arbeit.
Verdeutlichen Sie, warum das Thema relevant ist und welche aktuellen
Entwicklungen es gibt.
  
\item Darstellung des Problems

Beschreiben Sie das konkrete Problem, das Sie in Ihrer Arbeit untersuchen
werden. Erklären Sie, warum dieses Problem wichtig ist und welche negativen
Folgen es hat, wenn es nicht gelöst wird. Nutzen Sie konkrete Beispiele oder
Daten, um das Problem zu veranschaulichen.
  
\item Kontextualisierung:

Setzen Sie das Problem in den Kontext der bestehenden Forschung und Praxis.
Zeigen Sie auf, welche Lösungsansätze es bisher gab und warum diese unzureichend
sind.
\end{enumerate}



\section{Zielsetzung}

\subsection*{Zweck}

Der Abschnitt "Zielsetzung" definiert klar und präzise, was mit der Arbeit
erreicht werden soll. Er gibt dem Leser eine Vorstellung davon, welche
Ergebnisse am Ende erwartet werden.

\subsection*{Inhalt}

\begin{enumerate}

\item Definition der Ziele

Beschreiben Sie klar und präzise, welche Ziele mit der Arbeit verfolgt werden.
Differenzieren Sie zwischen Hauptzielen und Nebenzielen, falls relevant.

\item Nutzen und Relevanz

Erklären Sie, welchen Nutzen die Ergebnisse der Arbeit haben werden.
Verdeutlichen Sie, wie die Arbeit zur Lösung des in der Problemstellung
beschriebenen Problems beitragen wird.

\item Methoden und Ansatz

Skizzieren Sie kurz die Vorgehensweise und Methoden, die zur Erreichung der
Ziele eingesetzt werden. Erwähnen Sie wichtige Schritte wie Analysen,
Experimente, Evaluationen oder Implementierungen.
\end{enumerate}