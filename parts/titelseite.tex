%%% Schriftgröße SuperHuge definieren
%%% Die erste Zahl (36) ist die Schriftgröße in Punkten,
%%% die zweite Zahl (40) ist der Zeilenabstand.
\newcommand{\SuperHuge}{\fontsize{36}{40}\selectfont}

%%% Beginn der Titelseite
\begin{titlepage}

  %%% Hintergrund oben links

  % % Variante 1
  % \begin{tikzpicture}[remember picture, overlay]
  %   \node[anchor=north west] at (current page.north west) {
  %     \includegraphics[width=10cm]{figures/background-08}
  %   };
  % \end{tikzpicture}

  % Variante 2
  % \begin{tikzpicture}[remember picture, overlay]
  %   \node[anchor=north west, rotate=-20, xshift=-1.5cm, yshift=4.5cm] at (current page.north west) {
  %     \includegraphics[width=10cm]{figures/background-02}
  %   };
  % \end{tikzpicture}

  % Variante 3
  \begin{tikzpicture}[remember picture, overlay]
    \node[anchor=north west] at (current page.north west) {
      \includegraphics[width=12cm]{figures/background-09}
    };
  \end{tikzpicture}
  

  %%% Logo oben rechts
  \begin{tikzpicture}[remember picture, overlay]
    \node[anchor=north east, xshift=-12mm, yshift=-6mm] at (current page.north east) {
      \includegraphics[width=6cm]{figures/hs-esslingen-logo}
    };
  \end{tikzpicture}

  \centering

  \vspace*{3cm}

  %%% Titel oben (Variante 1)
  %{\SuperHuge \thetitle\par}  % Wert für \thetitle findet sich in metadaten.tex

  %%% Titel oben (Variante 2, dunkelblau hinterlegter Kasten und grauer Rand)
  %%%%%%%%%%%%%%%%%%%%%%%%%%%%%%%%%%%%%%%%%%%%%%%%%%%%%%%%%%%%%%%%%%%%%%%%%%%%%
  \sffamily

  \begin{tikzpicture}[remember picture, overlay]
    % Text-Node mit automatischem Umbruch
    \node[
        text=white,
        align=right,
        text width=14cm,
        font=\SuperHuge\sffamily\selectfont,
        anchor=north east,
        inner sep=1cm,
        outer sep=0pt
    ] (title) at ([xshift=-0.7cm,yshift=-0.25\paperheight]current page.north east) 
    {\thetitle\par};

    % Dunkelblauer Kasten, der sich an den Text anpasst und am rechten Rand beginnt
    \begin{scope}[on background layer]
        \fill[hse-dunkelblau] 
            ([yshift=-0.25\paperheight]current page.north east) rectangle 
            ([xshift=-14.7cm]current page.north east |- title.south);
    \end{scope}
    
    % Graue Linie am Rand (unverändert)
    \draw[hse-hellgrau,line width=5mm] 
        ([xshift=2mm,yshift=-1mm]current page.south west) rectangle 
        ([xshift=-2mm,yshift=1mm]current page.north east);
  \end{tikzpicture}

  \vspace*{5cm}
  %%%%%%%%%%%%%%%%%%%%%%%%%%%%%%%%%%%%%%%%%%%%%%%%%%%%%%%%%%%%%%%%%%%%%%%%%%%%%

  \vfill

  %%% Informationen in der Mitte
  {\large zur Erlangung des akademischen Grades eines\\}
  \vspace*{0.3cm}
  {\LARGE Bachelor of Engineering (B.\,Eng.)\\}
  \vspace*{0.8cm}
  {\large von der Fakultät für Mobilität und Technik\\
  der Hochschule Esslingen genehmigte\\}
  \vspace*{0.2cm}
  {\LARGE Bachelorarbeit\\}
  \vspace*{0.1cm}
  {\large von\\}
  \vspace*{0.06cm}
  {\LARGE \theauthor\\}   % Wert für \theauthor findet sich in metadaten.tex
  \vfill

  %%% Tabelle unten
  {\large
  \begin{tabular}{ll}
  Tag der Einreichung: & \thedate\\ % Wert für \thedate findet sich in metadaten.tex
  Erster Gutachter: & Prof.\ Dr.\ Martin Röhricht\\
  Zweiter Gutachter: & Vorname Nachname\\
  \end{tabular}}

\end{titlepage}
%%% Ende der Titelseite
