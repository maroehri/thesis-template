%%% Hier bitte die richtigen Werte einsetzen:
\newcommand{\CompanyName}{}         % Standard leer
\newcommand{\ConfidentialDate}{}    % Standard leer
\newcommand{\SubmissionDate}{}      % Standard leer

% Definition der Fallback-Werte
\newcommand{\FallbackCompanyName}{\colorbox{hse-hellblau}{\textcolor{white}{Thesis Inc.}}}
\newcommand{\FallbackConfidentialDate}{\colorbox{hse-hellblau}{\textcolor{white}{DD.MM.YYYY}}}
\newcommand{\FallbackSubmissionDate}{\colorbox{hse-hellblau}{\textcolor{white}{DD.MM.YYYY}}}

% Definition der zu verwendenden Werte
\newcommand{\UseCompanyName}{\ifthenelse{\equal{\CompanyName}{}}{\FallbackCompanyName}{\CompanyName}}
\newcommand{\UseConfidentialDate}{\ifthenelse{\equal{\ConfidentialDate}{}}{\FallbackConfidentialDate}{\ConfidentialDate}}
\newcommand{\UseSubmissionDate}{\ifthenelse{\equal{\SubmissionDate}{}}{\FallbackSubmissionDate}{\SubmissionDate}}

\thispagestyle{empty}

\vspace*{\fill}

\section*{Sperrvermerk}

Die nachfolgende Arbeit enthält vertrauliche Daten der Firma \UseCompanyName.
Veröffentlichungen oder Vervielfältigungen dieser Arbeit – auch nur auszugsweise
– sind ohne ausdrückliche Genehmigung der \UseCompanyName nicht gestattet. Diese
Arbeit ist nur den Prüfern sowie den Mitgliedern des Prüfungsausschusses
zugänglich zu machen. Der Sperrvermerk gilt bis zum \UseConfidentialDate. Danach
kann diese Arbeit frei veröffentlicht werden.

\vspace*{3cm}

\section*{Eigenständigkeitserklärung}

Hiermit versichere ich, die vorliegende Arbeit selbstständig und unter
ausschließlicher Verwendung der angegebenen Literatur und Hilfsmittel erstellt
zu haben.

\noindent Die Arbeit wurde bisher in gleicher oder ähnlicher Form keiner anderen
Prüfungsbehörde vorgelegt und auch nicht veröffentlicht.

\vspace*{2\baselineskip}
\hbox to \textwidth{\hrulefill}
\par
\noindent Esslingen, den \UseSubmissionDate
