\section*{Textformatierungen in LaTeX}

In \LaTeX\ gibt es verschiedene Befehle zur Textformatierung, die für
unterschiedliche Zwecke verwendet werden können. Diese Befehle helfen dabei, den
Text strukturiert und leserfreundlich zu gestalten. Hier sind einige wichtige
Befehle und ihre Anwendungen:

\subsection*{\texttt{\textbackslash textit\{\}} - Kursivschrift}

Der Befehl \texttt{\textbackslash textit\{\}} wird verwendet, um Text kursiv
darzustellen. Kursivschrift eignet sich besonders gut für Hervorhebungen
innerhalb des Fließtextes, wie z.B. für fremdsprachige Ausdrücke, Titel von
Büchern oder betonte Wörter.

\begin{itemize}
    \item Beispiel: \textit{Dies ist ein kursiver Text.}
\end{itemize}

\subsection*{\texttt{\textbackslash textsf\{\}} - Serifenlose Schrift}

Der Befehl \texttt{\textbackslash textsf\{\}} wird verwendet, um Text in
serifenloser Schrift darzustellen. Diese Schriftart eignet sich besonders gut
für feststehende Begriffe aus der jeweiligen Domäne oder für Begriffe, die sich
von normalem Fließtext abheben sollen.

\begin{itemize}
    \item Beispiel: \textsf{Dies ist ein Text in serifenloser Schrift.}
\end{itemize}

\subsection*{\texttt{\textbackslash textbf\{\}} - Fettschrift}

Der Befehl \texttt{\textbackslash textbf\{\}} wird verwendet, um Text fett
darzustellen. Fettschrift sollte sparsam verwendet werden und eignet sich
besonders gut für starke Hervorhebungen, wie z.B. wichtige Begriffe oder
Überschriften.

\begin{itemize}
    \item Beispiel: \textbf{Dies ist ein fetter Text.}
\end{itemize}

\subsection*{\texttt{\textbackslash texttt\{\}} - Monospace-Schrift}

Der Befehl \texttt{\textbackslash texttt\{\}} wird verwendet, um Text in einer
Monospace-Schrift (Schreibmaschinenschrift) darzustellen. Diese Schriftart
eignet sich besonders gut für Code, Programmnamen, Dateinamen und andere
technische Begriffe.

\begin{itemize}
    \item Beispiel: \texttt{Dies ist ein Text in Monospace-Schrift.}
\end{itemize}



\section*{Aufzählungen}

Aufzählungen sind eine häufig genutzte Methode, um Listen von Elementen
strukturiert darzustellen. In \LaTeX\ gibt es mehrere Möglichkeiten,
Aufzählungen zu erstellen. Zwei der häufigsten Methoden sind nummerierte
Aufzählungen mit \texttt{enumerate} und nicht nummerierte Aufzählungen mit
\texttt{itemize}.

\subsection*{Nummerierte Aufzählungen}

Nummerierte Aufzählungen verwenden Zahlen oder Buchstaben, um die Reihenfolge
der Elemente anzuzeigen. Dies ist besonders nützlich, wenn die Reihenfolge der
Elemente wichtig ist oder wenn Sie auf bestimmte Punkte verweisen möchten.

\begin{enumerate}
  \item Erster Punkt der Aufzählung.
  \item Zweiter Punkt der Aufzählung.
  \begin{enumerate}
      \item Unterpunkt 1 der zweiten Ebene.
      \item Unterpunkt 2 der zweiten Ebene.
      \begin{enumerate}
          \item Unterpunkt 1 der dritten Ebene.
          \item Unterpunkt 2 der dritten Ebene.
      \end{enumerate}
  \end{enumerate}
  \item Dritter Punkt der Aufzählung.
\end{enumerate}

\subsection*{Nicht nummerierte Aufzählungen}

Nicht nummerierte Aufzählungen verwenden Symbole wie Punkte oder Striche, um die
Elemente der Liste zu kennzeichnen. Dies ist nützlich, wenn die Reihenfolge der
Elemente keine Rolle spielt.

\begin{itemize}
  \item Erster Punkt der Aufzählung.
  \item Zweiter Punkt der Aufzählung.
  \begin{enumerate}
      \item Unterpunkt 1 der zweiten Ebene.
      \item Unterpunkt 2 der zweiten Ebene.
      \begin{enumerate}
          \item Unterpunkt 1 der dritten Ebene.
          \item Unterpunkt 2 der dritten Ebene.
      \end{enumerate}
  \end{enumerate}
  \item Dritter Punkt der Aufzählung.
\end{itemize}


\section*{Literaturangaben}

In der Datei \texttt{thesis.bib} befinden sich die Literaturangaben, die im
folgenden genutzt werden:

In der heutigen Forschung sind viele verschiedene Quellenarten relevant. Ein
grundlegendes Werk im Bereich der Textsatzsysteme ist
\textcite{knuth1984texbook}. Auch die Relativitätstheorie von
\textcite{einstein1905} spielt eine bedeutende Rolle in der Physik.

Die Grundlagen moderner Textsatzsysteme wurden maßgeblich durch die Arbeiten von
\textcite{lamport1986document} geprägt. Auch die Definition des HTTP-Protokolls
ist von großer Bedeutung \cite{rfc2616}.

In der Informatik sind Algorithmen ein wichtiger Bestandteil der Forschung.
Hierzu zählen auch die grundlegenden Arbeiten von \textcite{knuth1997art}.
Turingmaschinen, wie sie in der Dissertation von \textcite{hopcroft1971thesis}
beschrieben werden, sind ein weiteres zentrales Thema.

Ein sehr interessantes Buch zum Thema "Software Architektur" ist von
\textcite{headFirstSoftwareArchitecture} erschienen.

Technische Berichte sind oft die Basis für neue Entwicklungen. Ein Beispiel
hierfür ist der Bericht von \textcite{lamport1994latex} über LaTeX.

Die Arbeiten von \textcite{turing1936thesis} haben die Forschung im Bereich der
theoretischen Informatik maßgeblich beeinflusst. Zudem bietet die Webseite des
\textcite{latexproject} wertvolle Informationen für Anwender.

Abschließend soll noch ein weiteres Beispiel für ein Buch genannt werden, das
gesellschaftliche und bildungspolitische Aspekte beleuchtet:
\textcite{golden2019book}.



\section*{Nutzung von Farben}

In \LaTeX\ können Farben verwendet werden, um Text, Tabellen und andere Elemente
hervorzuheben und das Dokument ansprechender zu gestalten. Hierfür ist in
unserer Vorlage bereits das \texttt{xcolor}-Paket eingebunden, das eine Vielzahl
von Farbmodellen unterstützt, einschließlich RGB, CMYK und HTML-Farben.

Folgende Farben sind bereits vordefiniert und können über den unten angegebenen
Farbcode verwendet werden:

\begin{table}[htb!]
  \centering
  \caption{Tabelle mit Farbbegriffen und entsprechenden Farben}
  \begin{tblr}{
    colspec = {X[c,m] X[c,m]},
    stretch=1.15,
    column{1} = {font=\ttfamily},
    column{2} = {font=\sffamily},
    row{1} = {font=\bfseries\sffamily},
    cell{2}{2} = {hse-dunkelblau, fg=white},
    cell{3}{2} = {hse-rot, fg=white},
    cell{4}{2} = {hse-hellblau, fg=white},
    cell{5}{2} = {hse-blau75, fg=white},
    cell{6}{2} = {hse-blau50, fg=white},
    cell{7}{2} = {hse-blau25, fg=hse-dunkelblau},
    cell{8}{2} = {hse-blau15, fg=hse-dunkelblau},
    cell{9}{2} = {hse-hellgrau, fg=hse-dunkelblau},
    cell{10}{2} = {mittelgrau, fg=hse-dunkelblau},
    cell{11}{2} = {dunkelgrau, fg=white},
  }
  \toprule
  Farbcode       & Farbe \\ \midrule
  hse-dunkelblau & HS Esslingen Dunkelblau \\
  hse-rot        & HS Esslingen Rot \\
  hse-hellblau   & HS Esslingen Hellblau \\
  hse-blau75     & HS Esslingen Blau 75\% \\
  hse-blau50     & HS Esslingen Blau 50\% \\
  hse-blau25     & HS Esslingen Blau 25\% \\
  hse-blau15     & HS Esslingen Blau 15\% \\
  hse-hellgrau   & HS Esslingen Hellgrau \\
  mittelgrau     & Mittelgrau \\
  dunkelgrau     & Dunkelgrau \\
  \bottomrule
  \end{tblr}
\end{table}


\noindent Sie können so die Farben wie folgt verwenden: \texttt{\textbackslash
textcolor\{Farbe\}\{Text\}} färbt den Text in der angegebenen Farbe:

\begin{itemize}
  \item \textcolor{hse-rot}{Dies ist ein Text in Rot.}
\end{itemize}

\noindent \texttt{\textbackslash colorbox\{Farbe\}\{Text\}} setzt den Hintergrund des
Textes in der angegebenen Farbe.

\begin{itemize}
  \item \colorbox{hse-blau15}{Dies ist ein Text mit sehr hellblauem Hintergrund.}
\end{itemize}



\section*{Einbinden von Grafiken}

Grafiken können wie folgt eingebunden werden, hier zu sehen am Beispiel der
Abbildung~\ref{fig:example}:

%%% Einfügen einer Abbildung mit spezifischen Platzierungsoptionen:
% "h" steht für "here": versucht, die Abbildung an der aktuellen Position im
% Text einzufügen
% "t" steht für "top": versucht, die Abbildung am oberen Rand der Seite
% einzufügen
% "b" steht für "bottom": versucht, die Abbildung am unteren Rand der Seite
% einzufügen
% "!" steht für "override" und erzwingt eine strikte Platzierung, ignoriert
% einige der internen LaTeX-Einschränkungen zur Platzierung von Abbildungen
\begin{figure}[htb!]
  \centering
  \includegraphics[width=0.5\textwidth]{figures/hs-esslingen-logo}
  \caption{Beispiel für die Einbindung einer PDF-Grafik mit Beschriftung.}
  \label{fig:example}
\end{figure}

Hier ist ein Beispiel, wie zwei Bilder nebeneinander eingefügt werden können,
mit einer gemeinsamen Beschriftung für beide Bilder (siehe
Abbildung~\ref{fig:combined}) und separaten Unterüberschriften für jedes Bild
(Abbildung~\ref{fig:example1} und Abbildung~\ref{fig:example2}).

\begin{figure}[htb!]
  \centering
  \begin{minipage}[t]{0.45\textwidth}
    \centering
    \includegraphics[width=\textwidth]{figures/hs-esslingen-logo}
    \subcaption{Erste Unterüberschrift}
    \label{fig:example1}
  \end{minipage}%
  \hfill
  \begin{minipage}[t]{0.45\textwidth}
    \centering
    \includegraphics[width=\textwidth]{figures/hs-esslingen-logo}
    \subcaption{Zweite Unterüberschrift}
    \label{fig:example2}
  \end{minipage}
  \caption{Gemeinsame Beschriftung für beide Bilder}
  \label{fig:combined}
\end{figure}





\section*{Tabellen}

Tabellen in \LaTeX\ waren und sind teilweise immer noch schwieriger zu
gestalten, als in anderen Programmen. Es gibt mittlerweile ein Paket
\texttt{tabularray}, welches die Handhabung sehr erleichtert und viele
Konfigurationsmöglichkeiten bietet.

\begin{table}[htb!]
  \centering
  \caption{Überprüfung von Aussagen zur Softwareentwicklung}
  \label{tab:true-false-tabelle}
\begin{tblr}{
  colspec = {Q[c,m] X[l,m] Q[c,m] Q[c,m]},
  row{odd}={bg=hellgrau},   % Schattierung der ungeraden Zeilen
  row{1}={bg=mittelgrau, font=\bfseries\sffamily},  % Überschrift
  stretch=1.5               % Vergrößerung des Zeilenabstands
}
\toprule
Nr. & Aussage & Wahr & Falsch\\ \midrule
(1) & Da in den frühen Phasen der Softwareentwicklung kein Code verfügbar ist, sind manuelle Überprüfungen nicht anwendbar.
        & {\Large $\Box$} & {\Large $\boxtimes$} \\
(2) & Code-Reviews sollten nur von Personen durchgeführt werden, die nicht Teil des Entwicklungsteams sind.
        & {\Large $\Box$} & {\Large $\boxtimes$} \\
(3) & Paarprogrammierung kann auch eine Form der Code-Überprüfung sein.
        & {\Large $\boxtimes$} & {\Large $\Box$} \\
(4) & Code-Reviews sollten immer von einer einzelnen Person durchgeführt werden, um Konsistenz und klare Verantwortlichkeiten zu gewährleisten.
        & {\Large $\Box$} & {\Large $\boxtimes$} \\
\bottomrule
\end{tblr}
\end{table}

\subsection*{Beispiel für zwei nebeneinander stehende Tabellen}

Falls Sie zwei Tabellen nebeneinander platzieren wollen, können Sie wie bei den
Abbildungen zwei \texttt{minipage}s anlegen.

\begin{table}[htb!]
  \centering
  \begin{minipage}[t]{.45\textwidth}
    \centering
    \caption{Erste Überprüfungen}
    \begin{tblr}{
      colspec = {Q[c,m] X[l,m] Q[c,m] Q[c,m]},
      row{odd}={bg=hellgrau},   % Schattierung der ungeraden Zeilen
      row{1}={bg=mittelgrau, font=\bfseries\sffamily},  % Überschrift
      stretch=1.5               % Vergrößerung des Zeilenabstands
    }
    \toprule
    Nr. & Aussage & Wahr & Falsch\\ \midrule
    (1) & Aussage 1 & {\Large $\Box$} & {\Large $\boxtimes$} \\
    (2) & Aussage 2 & {\Large $\Box$} & {\Large $\boxtimes$} \\
    \bottomrule
    \end{tblr}
  \end{minipage}%
  \hfill
  \begin{minipage}[t]{.45\textwidth}
    \centering
    \caption{Zweite Überprüfungen}
    \begin{tblr}{
      colspec = {Q[c,m] X[l,m] Q[c,m] Q[c,m]},
      row{odd}={bg=hellgrau},   % Schattierung der ungeraden Zeilen
      row{1}={bg=mittelgrau, font=\bfseries\sffamily},  % Überschrift
      stretch=1.5               % Vergrößerung des Zeilenabstands
    }
    \toprule
    Nr. & Aussage & Wahr & Falsch\\ \midrule
    (1) & Aussage 3 & {\Large $\Box$} & {\Large $\boxtimes$} \\
    (2) & Aussage 4 & {\Large $\boxtimes$} & {\Large $\Box$} \\
    \bottomrule
    \end{tblr}
  \end{minipage}
\end{table}



\section*{Verwendung von Mathematik in \LaTeX}

\LaTeX{} ist ein leistungsfähiges Textsatzsystem, das sich besonders gut für das
Schreiben von wissenschaftlichen Arbeiten eignet. Ein wesentliches Merkmal von
\LaTeX{} ist seine Fähigkeit, komplexe mathematische Ausdrücke sauber und
präzise zu setzen. In diesem Abschnitt wird die Verwendung von Mathematik in
\LaTeX{} kurz erläutert.

\subsection*{Inline-Mathematik}

Mathematische Ausdrücke können innerhalb eines Textes eingefügt werden, indem
man sie zwischen Dollarzeichen setzt. Zum Beispiel wird die Quadratformel wie
folgt geschrieben: \( ax^2 + bx + c = 0 \).

\subsection*{Abgesetzte Mathematik}

Für größere oder wichtigere Ausdrücke verwendet man abgesetzte Mathematik. Diese
Ausdrücke werden zentriert und auf einer eigenen Zeile dargestellt. Man
verwendet dazu die \texttt{\textbackslash[ \dots \textbackslash]}-Umgebung:

\[
e^{i\pi} + 1 = 0
\]

oder die \texttt{equation}-Umgebung, die auch eine Gleichungsnummer hinzufügt:

\begin{equation}
e^{i\pi} + 1 = 0
\end{equation}

\subsection*{Mehrere Gleichungen mit \texttt{align}}

Die \texttt{align}-Umgebung ist besonders nützlich, um mehrere Gleichungen
auszurichten. Jede Gleichung wird durch ein \texttt{\&}-Zeichen ausgerichtet und
durch ein \texttt{\textbackslash\textbackslash} beendet. Hier ist ein
komplexeres Beispiel:

\begin{align}
a + b &= c \\
d + e &= f \\
x_1 + y_1 &= z_1 \\
2x_2 + 3y_2 &= 5z_2
\end{align}

\section*{Mathematische Symbole und Operatoren}

\LaTeX{} bietet eine breite Palette von mathematischen Symbolen und Operatoren.
Einige häufig verwendete Symbole sind:

\begin{itemize}
    \item Summenzeichen: \(\sum_{i=1}^n i\)
    \item Integralzeichen: \(\int_0^1 x^2 \, dx\)
    \item Brüche: \(\frac{a}{b}\)
    \item Wurzeln: \(\sqrt{x}\) und \(\sqrt[3]{x}\)
\end{itemize}

\subsection*{Griechische Buchstaben}

Griechische Buchstaben werden häufig in mathematischen Ausdrücken verwendet.
Hier sind einige Beispiele:

\begin{itemize}
    \item Kleinbuchstaben: \(\alpha, \beta, \gamma, \delta, \epsilon\)
    \item Großbuchstaben: \(\Gamma, \Delta, \Theta, \Lambda, \Pi, \Sigma, \Phi,
    \Psi, \Omega\)
\end{itemize}

\subsection*{Vektoren und Matrizen}

Vektoren und Matrizen können ebenfalls in \LaTeX{} dargestellt werden:

\begin{itemize}
    \item Vektoren: \(\vec{v}\) oder \(\mathbf{v}\)
    \item Matrizen: 
    \begin{equation}
    \mathbf{A} = \begin{pmatrix}
    a & b \\
    c & d
    \end{pmatrix}
    \end{equation}
\end{itemize}



\section*{Die mybox-Umgebung}

Demonstration der Nutzung einer \texttt{mybox}-Umgebung:

\begin{mybox}[Bildschirmausgabe]{console}
  Kleinstes Element: 2
\end{mybox}

Sollen einzelne Elemente in dieser Umgebung fett hervorgehoben werden, so nutzt
man dazu die Anweisung \texttt{\textbackslash highlight\{\}}

\begin{mybox}[Textdatei.txt]{file}
  \highlight{5} 24 13 83 22 4 \highlight{3} 99 23 45 \highlight{4} 82 34 11 9 \highlight{6} 13 22 93 42 85 34
\end{mybox}


\section*{Source Code}

Falls mit Quelltext und Syntax-Highlighting gearbeitet werden soll, empfehle ich
die Verwendung des \texttt{minted}-Pakets. Dieses setzt allerdings folgendes
voraus:

\begin{itemize}
  \item Python muss auf Ihrem System installiert sein.
  \item Pygments muss ebenfalls installiert sein. Dies kann durch Ausführen von
  \texttt{pip install pygments} erreicht werden.
  \item Da \texttt{minted} externe Programme aufruft, muss LaTeX mit der Option
  \texttt{-shell-escape} kompiliert werden, um diese Aufrufe zu erlauben. Das
  wäre in unserem enthaltenen \texttt{Makefile} bereits der Fall.
\end{itemize}

\begin{minted}{c}
  #include <stdio.h>
  
  int main(void) {
      char string[] = "Fischers Fritze fischt frische Fische, frische Fische fischt Fischers Fritze";
      char word[] = "Fische";
  
      // Die im Folgenden genutzte countWords()-Funktion
      // müssen Sie erstellen
      int count = countWords(string, word);
  
      printf("Das Wort \"%s\" kam %d-mal im String vor", word, count);
  }
\end{minted}

Hier noch ein Beispiel, bei dem die Zeilen 4-5 und 11 mittels der Option
\texttt{hightlightlines} mit einer Hintergrundfarbe hervorgehoben wurden und die
Zeilennummern mittels der Option \texttt{linenos} aktiviert wurden.


\begin{minted}[linenos,highlightlines={4-5,11}, highlightcolor=hse-blau15]{c}
  #include <stdio.h>
  
  int main(void) {
      char string[] = "Fischers Fritze fischt frische Fische, frische Fische fischt Fischers Fritze";
      char word[] = "Fische";
  
      // Die im Folgenden genutzte countWords()-Funktion
      // müssen Sie erstellen
      int count = countWords(string, word);
  
      printf("Das Wort \"%s\" kam %d-mal im String vor", word, count);
  }
\end{minted}



\section*{ToDo-Notizen}

Das \texttt{todonotes}-Paket bietet eine einfache Möglichkeit, Aufgaben und
Notizen in \LaTeX\-Dokumenten zu verwalten. Es erlaubt Ihnen, ToDo-Notizen
sowohl im Randbereich als auch direkt im Text einzufügen und sogar Platzhalter
für fehlende Abbildungen zu setzen.

Mit dem \texttt{\textbackslash todo}-Befehl können Sie Notizen im Randbereich
Ihres Dokuments einfügen, zum Beispiel genau an diese\todo{Hier eine ToDo-Notiz
im Randbereich.} Stelle. Dies ist besonders nützlich für kurze Erinnerungen oder
Hinweise.

Manchmal ist es praktischer, die Notizen direkt im Text anzuzeigen. Das
ermöglicht der \texttt{\textbackslash todo[inline]\{\}}-Befehl.

\todo[inline]{Dies ist eine inline ToDo-Notiz im Text.}

Mit dem \texttt{\textbackslash missingfigure\{\}}-Befehl können Sie Platzhalter
für fehlende Abbildungen setzen. Dies ist nützlich, um zu kennzeichnen, wo
später noch Abbildungen eingefügt werden sollen.

\missingfigure{Eine Skizze der Struktur eines Trebuchets einfügen.}
